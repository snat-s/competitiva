% Created 2022-03-29 Tue 18:31
% Intended LaTeX compiler: pdflatex
\documentclass[11pt]{article}
\usepackage[utf8]{inputenc}
\usepackage[T1]{fontenc}
\usepackage{graphicx}
\usepackage{grffile}
\usepackage{longtable}
\usepackage{wrapfig}
\usepackage{rotating}
\usepackage[normalem]{ulem}
\usepackage{amsmath}
\usepackage{textcomp}
\usepackage{amssymb}
\usepackage{capt-of}
\usepackage{hyperref}
\usepackage{color}
\usepackage{listings}
\author{Santiago Pedroza Díaz}
\date{\today}
\title{Algoritmario}
\hypersetup{
 pdfauthor={Santiago Pedroza Díaz},
 pdftitle={Algoritmario},
 pdfkeywords={},
 pdfsubject={},
 pdfcreator={Emacs 27.2 (Org mode 9.4.4)}, 
 pdflang={English}}
\begin{document}

\maketitle
\tableofcontents


\section{Introducción}
\label{sec:org92cb10b}

Este es el algoritmario que e estado haciendo conforme a pasado el tiempo.
Si utilizas \texttt{Emacs} puedes exportar a \texttt{pdf} o \texttt{epub} si son más cómodos que una página
web.

\section{Estructuras de Datos de mortales y medio mortales}
\label{sec:orgf2092aa}

\subsection{Clásicas}
\label{sec:org2166eb8}

\subsubsection{Stack}
\label{sec:org0c90c3b}

\subsubsection{Queue}
\label{sec:org85f1cd5}

\subsubsection{Linked List}
\label{sec:org9e79f8c}

\subsection{Logarítmicas}
\label{sec:org3768528}

\subsubsection{Priority Queue}
\label{sec:orgb34a23e}

\subsubsection{Multiset}
\label{sec:orgd1d89e6}

\section{Gráficas}
\label{sec:orgbe5c12b}
\subsection{Representaciónes de Gráficas}
\label{sec:org81d6ea2}
\subsubsection{Matrices de adyacencia}
\label{sec:orgced0394}

\subsubsection{Matrices de}
\label{sec:org0e96fce}

\subsection{BFS y DFS}
\label{sec:org964fd18}
\subsubsection{DFS}
\label{sec:org64fa341}

\subsubsection{BFS}
\label{sec:org842a5ee}
\subsection{Disjointed Set Union (DSU)}
\label{sec:org0184f43}
\subsubsection{Union by Rank}
\label{sec:org21a521b}

Funcion en tiempo \(O(log(N))\) por operación.

\lstset{language=C++,label= ,caption= ,captionpos=b,numbers=none}
\begin{lstlisting}
struct DSU {
	vector<int> P, H;
	DSU(int n) : P(n), H(n, 0) {
		for(int i = 0; i < n; i++)
			P[i] = i;
	}
	int find(int x) {
		if(P[x] == x) {
			return x;
		}
		return find(P[x]);
	}
	void merge(int x, int y) {
		int RootX = find(x);
		int RootY = finx(y);
		if (H[RootX] > H[RootY]) {
			swap(rootX, rootY);
		}
		P[RootX] = RootY;
		if (H[RootX] == H[RootY]) {
			H[RootY]++;
		}
	}
}
\end{lstlisting}
\subsubsection{Path Compression}
\label{sec:org5e0c0c8}

Es bastante rápida y en la práctica si quieres implementar algo desde
cero puedes llegar a usarla. Corre en tiempo \(O(log(N))\) amortizado.

\lstset{language=C++,label= ,caption= ,captionpos=b,numbers=none}
\begin{lstlisting}
struct DSU {
	vector<int> P;
	DSU(int n) : P(n) {
		for(int i = 0; i < n; i++) {
			P[i] = i;
		}
	}
	int find(int x) {
		if (P[x] == x) {
			return x;
		}
		return P[x] = find(P[x]);
	}
	void merge(int x, int y) {
		P[find(x)] = find(y);
	}
}
\end{lstlisting}
\subsubsection{Union by Rank y Path Compression}
\label{sec:org53fd00b}

Funciona en \(O(\alpha(N))\) amortizado y siendo realista es la única que vas a necesitar.

\(\alpha\): es la función inversa de Ackerman, es casi \(O(1)\).

\lstset{language=C++,label= ,caption= ,captionpos=b,numbers=none}
\begin{lstlisting}
struct DSU {
	vector<int> P, H;
	DSU(int n) : P(n), H(n, 0) {
		for (int i = 0; i < n; ++i) {
			P[i] = i;
		}
	}
	int find(int x) {
		if (P[x] == x) {
			return x;
		}
		return P[x] = find(P[x]);
	}
	void merge(int x, int y) {
		int rx = find(x);
		int ry = find(y);
		if(H[rx] > H[ry])
			swap(rx, ry);
		P[rx] = ry;
		if (H[rx] == H[ry]) {
			H[ry]++;
		}
	}
	bool same(int a, int b) {
		return find(a) == find(b);
	}
	void print() {
		for(auto x : P) {
			cout << x << " ";
		}
		std::cout << "\n";
	}
};
\end{lstlisting}

\subsection{MST}
\label{sec:org1f8e89b}
\subsubsection{Kruskal}
\label{sec:org4a3af96}

Vamos a utilizar un DSU con una ligera modificación,
ahora vamos a ponerle same.

\lstset{language=C++,label= ,caption= ,captionpos=b,numbers=none}
\begin{lstlisting}
struct DSU {
	// DSU de la parte anterior
	bool same(int x, int y) {
		return find(x) == find(y);
	}
};
\end{lstlisting}

El código de Kruskal en:
\lstset{language=C++,label= ,caption= ,captionpos=b,numbers=none}
\begin{lstlisting}
priority_queue<tuple<int,int,int>> pq; // w, a, b
int a, b, c;
for (int i = 0; i < m; ++i) {
	cin >> a >> b >> c;
	pq.push({-1*c,a,b});
 }
DSU dsu(n+1);
long long ans = 0;
int size = 0;
while (!pq.empty()) {
	int w, a, b;
	tie(w,a,b) = pq.top();
	pq.pop();
	if (!dsu.same(a, b)) {
		size = max(dsu.merge(a, b), size);
		cout << w*-1 << "\n";
		ans += (w*-1);
	}
 }
\end{lstlisting}

\subsubsection{Prim}
\label{sec:org4482d32}

\lstset{language=C++,label= ,caption= ,captionpos=b,numbers=none}
\begin{lstlisting}
ll prim(int start, vector<bool> &visited, const vector<vector<pair<int, int>>> &adj) {
	priority_queue<tuple<int, int>> pq;
	ll ans = 0, counter = 1;
	visited[start] = true;
	for(auto a : adj[start]) {
		pq.push({-1*a.second, a.first});
	}

	while (!pq.empty()) {
		int w, node;
		tie(w, node) = pq.top();
		pq.pop();
		if (visited[node]) continue;
		ans+=w*-1;
		counter++;
		visited[node] = true;
		for(auto a : adj[node]) {
			pq.push({a.second*-1,a.first});
		}
	}
	return  ans;
}
\end{lstlisting}
\subsection{Single Source Shortest Path}
\label{sec:org9d0c31a}
\subsubsection{Dijikstra}
\label{sec:orga8c5a81}
\lstset{language=C++,label= ,caption= ,captionpos=b,numbers=none}
\begin{lstlisting}
vector<vector<pair<ll,ll>>> adj(n+1); // b, w
priority_queue<tuple<ll, ll>> pq; // w, a
vector<ll> distance(n+1, INF);

ll a, b, c;
for (int i = 0; i < m; ++i) {
	cin >> a >> b >> c;
	adj[a].push_back({b,c});
 }

pq.push({0,1});
distance[1] = 0;

while (!pq.empty()) {
	ll w, a;
	tie(w, a) = pq.top();
	pq.pop();
	if(processed[a]) continue;
	processed[a] = true; 
	for(auto u : adj[a]) {
		ll b = u.first, w = u.second;
		if (distance[a]+w < distance[b]) {
			distance[b] = distance[a]+w;
			pq.push({-1*distance[b], b});
		}
	}
 }
\end{lstlisting}
\subsubsection{Bellman Ford}
\label{sec:org8bdc4bc}

\subsubsection{Floyd Warshall \(O(n^3)\)}
\label{sec:org8c8834f}

Sirve para cuando te preguntan varias veces la misma pregunta
de encontrar el camino más corto.

Usas una matriz de adyacencias.

\lstset{language=C++,label= ,caption= ,captionpos=b,numbers=none}
\begin{lstlisting}
// Inicializar matriz de adyacencias
for(int i = 0; i < n; i++) {
	for(int j = 0; j < n; j++) {
		if(i == j) mat[i][j] = 0;
		else mat[i][j] = INF;
	}
 }

// procesar la matriz de adyacencias y cambiar los valores
// de los caminos conocidos

// Floyd Warshall
for(int i = 0; i < n; i++) {
	for(int j = 0; j < n; j++) {
		for(int k = 0; k < n; k++) {
			mat[j][k]=min(mat[j][k], mat[j][i]+mat[i][k]);
		}
	}
 }
\end{lstlisting}

\section{Árboles}
\label{sec:org7b242bd}

\subsubsection{Árboles Binarios}
\label{sec:org3f81275}

\lstset{language=C++,label= ,caption= ,captionpos=b,numbers=none}
\begin{lstlisting}
struct node {
	int val;
	node *right, *left;
	node(int v) : val(v), right(nullptr), left(nullptr){}
	node() : val(0), right(nullptr), left(nullptr){}
};
\end{lstlisting}

\begin{enumerate}
\item Recorrer un árbol binario
\label{sec:org405f023}

Los códigos de inorden, orden y postorden tienen la misma idea de trasfondo.
La única diferencia es el momento en donde visitas el valor actual en el que
estás o lo dejas en el stack recursivo.

El recorrido por niveles es el único que cambia ya que en este caso tenemos
que hacer una exploración con una queue.

\begin{enumerate}
\item Inorden
\label{sec:org3bcbc77}

\lstset{language=C++,label= ,caption= ,captionpos=b,numbers=none}
\begin{lstlisting}
void inorden(node *root) {
	if(root == nullptr) return;
	cout << root->val << " ";
	inorden(root->left);
	inorden(root->right);
}
\end{lstlisting}

\item Orden
\label{sec:orgd7e2de9}

\lstset{language=C++,label= ,caption= ,captionpos=b,numbers=none}
\begin{lstlisting}
void orden(node *root) {
	if(root == nullptr) return;
	orden(root->left);
	cout << root->val << " ";
	orden(root->right);
}
\end{lstlisting}

\item Postorden
\label{sec:org3354989}

\lstset{language=C++,label= ,caption= ,captionpos=b,numbers=none}
\begin{lstlisting}
void postorden(node *root) {
	if(root == nullptr) return;
	postorden(root->left);
	postorden(root->right);
	cout << root->val << " ";
}
\end{lstlisting}

\item Orden por Nivel
\label{sec:orgdc3987d}

Una BFS.
\lstset{language=C++,label= ,caption= ,captionpos=b,numbers=none}
\begin{lstlisting}
queue<tuple<TreeNode*, int>> q;
vector<vector<int>> ans;
q.push({root, 0});
while(!q.empty()) {
	TreeNode* node;
	int CurrentDepth;
	tie(node, CurrentDepth) = q.front();
	q.pop();
	if(node == nullptr) continue;
	if(ans.size() <= CurrentDepth)
		ans.resize(CurrentDepth+1);
	ans[CurrentDepth].push_back(node->val);
	q.push({node->left, CurrentDepth+1});
	q.push({node->right, CurrentDepth+1});
 }
\end{lstlisting}
\end{enumerate}
\end{enumerate}
\subsubsection{Árboles Binarios de Búsqueda}
\label{sec:org65eed38}

\lstset{language=C++,label= ,caption= ,captionpos=b,numbers=none}
\begin{lstlisting}
class Node {
public:
	int val;
	Node *right, *left;

	Node(int v) : val(v), right(nullptr), left(nullptr){}
	Node() : val(0), right(nullptr), left(nullptr){}
};

Node* add(Node *root, int val) {
	if(root == nullptr) {
		return new Node(val);
	}
	if(root->val<val) {
		root->right = add(root->right, val);
	} else {
		root->left = add(root->left, val);
	}
	return root;
}
bool find(Node *root, int val) {
  if (root == nullptr) {
		return false;
  }
  if (root->val == val) {
		return true;
  }
	if(val > root->val) {
		return find(root->right, val);
	}

	return find(root->left,val);
}
\end{lstlisting}
\subsubsection{Árbol N-ario}
\label{sec:org6ddbaf2}
\lstset{language=C++,label= ,caption= ,captionpos=b,numbers=none}
\begin{lstlisting}
struct node {
	int val;
	vector<node*> children;
};
\end{lstlisting}

\section{Matemáticas}
\label{sec:orgbca84e3}

\subsection{Criba de Erastostenes}
\label{sec:org428ca0d}

\subsection{Exponenciación Binaria}
\label{sec:org0611e5c}

\subsection{Propiedades de los módulos}
\label{sec:org731f0cf}

\subsection{Supuesto Big Int en C++ sin que sea Big Int (o como usar strings)}
\label{sec:orga538b9f}
\end{document}